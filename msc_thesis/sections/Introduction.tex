\chapter{Introduction}
\label{ch:intro}


In a Graphics Processing Unit (GPU) program, 
also called a kernel, the \textit{kernel launch parameters}
specify how threads are mapped to,
and executed by, the hardware resources of the GPU. 
The choice of kernel launch parameters
has a profound impact on the running time
of a GPU program. 
Moreover, as we will show, 
the optimal choice of kernel launch parameters
is not fixed for a given GPU kernel. 
Rather, it depends on the values of other
\textit{data} and \textit{hardware parameters}.
In this thesis we present a generic technique, 
and an implementation of that technique specialized to CUDA, 
for automatically and dynamically determining 
kernel launch parameters
which minimize the running time of each kernel invocation in a CUDA program.

Generally, three types of parameters 
influence the performance of a parallel program: 
\begin{inparaenum}[(i)]
	\item {\em data parameters}, such as input data and its size;
	\item {\em hardware parameters}, such as cache capacity and number of 
	available registers; and
	\item {\em program parameters}, such as granularity of tasks and 
	the quantities that characterize how tasks are
	mapped to processors.
\end{inparaenum}
These parameters are independent of the parallel program itself.
Data and hardware parameters are independent from program parameters
and are determined by the needs of the user and available hardware
resources.  Program parameters, however, are intimately related to
data and hardware parameters. 
Therefore, it is crucial to determine values of program
parameters that yield the best program performance for a given set of
hardware and data parameter values.

The program parameters of interest to us 
in the {\cuda} programming model are thread block configurations. 
Their impact on performance is obvious 
considering, for example, that the memory access pattern of threads
in a thread block depend on the thread block's dimension sizes.
Similarly, a thread block configuration which 
minimizes the running time of a kernel invocation  
may not be optimal if the data sizes or target GPU device changes~\cite{DBLP:journals/tjs/TorresGL13}.
This emphasizes not only the impact of data and hardware parameters on program parameters, 
but also the need for performance portability.
Unfortunately, in practice, thread block configurations are typically 
determined statically for a kernel through simple heuristics or trial and error, 
so long as the configurations are constrained to be multiples of 32 \cite{cuda2016best}.


In most cases, the values of the data parameters are only given at
runtime, making it difficult to determine optimal values of the
program parameters at an earlier stage. On another hand, a bad choice
of program parameters can have drastic consequences.  Hence, it is
crucial to be able to determine the optimal program parameters at
runtime without much overhead added to the program execution. 
This is
precisely the intention of the approach proposed here. %in this paper.


%That is to say, enabling users to efficiently execute the same parallel program
%on different architectures that belong to the same hardware platform.


KLARAPTOR (Kernel LAunch
parameters RAtional Program estimaTOR)
is a tool for automatically and
dynamically determining the values of {\cuda} kernel launch parameters which
optimize the kernel's performance, for each kernel invocation
independently.  That is to say, based on the actual data and target
device of a kernel invocation.
%However, our tool utilizes rational 
%	programs to dynamically optimize thread block configurations based on the size 
%	of data being operated on, resulting in improved performance for specific kernel 
%	invocations on target architectures.}

KLARAPTOR consists of two main steps: 
\begin{inparaenum}[(i)]
	\item at compile-time, we determine formulas describing low-level performance metrics 
	for each kernel and insert those formulas into the host code of a CUDA program; and
	\item at runtime, a \textit{helper program} corresponding to a particular kernel evaluates
	those formulas using the actual data and hardware parameters to 
	determine the thread block configuration that minimizes the kernel's execution time. 
\end{inparaenum}

To minimize compilation and execution overheads, KLARAPTOR performs compile-time analysis and extrapolation 
of kernel performance on small data sizes at compile-time to predict kernel performance at runtime. 
Evaluation using the Polybench/GPU benchmark suite
demonstrates that KLARAPTOR accurately predicts near-optimal thread block configuration see Figure~\ref{fig:kernelperfintro}.


\begin{figure}[t!]
	\includegraphics[width=1.00\textwidth, trim={0em, 0.9em, 0em, 0em}, clip]{Figures/MinMaxTimes-RTX-8192.png}
	\caption{Comparing kernel execution time (log-scaled) for the thread block configuration chosen by KLARAPTOR
		versus the minimum and maximum times as determined by an exhaustive search over all possible configurations. 
		Kernels are part of the \texttt{PolyBench/GPU} benchmark suite and executed on a RTX 2070 SUPER with a data 
		size of $N=8192$ (except convolution3d with $N=512$)}
	\label{fig:kernelperfintro}
\end{figure}


Our key principle is based on an observation 
that, in most performance prediction models,
\textit{high-level performance metrics} (e.g. execution time, hardware occupancy)
can be seen as decision trees or flowcharts
based on \textit{low-level performance metrics} (e.g. memory bandwidth, cache miss rate).
%, and the number of clock cycles per instruction.
These low-level metrics are themselves
piece-wise rational functions (PWRFs) of
program, data, and hardware parameters.
%This construction could be applied
%recursively to obtain a PRF for the high-level metric.
%\fixed{We regard a computer program that computes such a PRF as a {\em rational program}}{I do not feel that there is a logical flow here},
%a technical notion defined in Section~\ref{sec:foundations}.
If one could determine these PRFs, 
then it would be possible to estimate, for example, 
the running time of a program based on its
program, data, and hardware parameters. 


Unfortunately, exact formulas for low-level metrics are often not known, 
instead estimated through empirical measures or
assumptions, or collected by profiling.
This is a key challenge our technique addresses.
%We introduce a generic technical notion describing these
%PRFs as \textit{rational programs} in Section~\ref{sec:foundations}.

\section{Contributions}
\label{sec:contributions}

The goal of this work is to determine values of program parameters which
optimize a multithreaded program's performance.  Towards that goal,
the method by which such values are found must be
receptive to changing data and changing  hardware parameters. Our contributions
encapsulate this requirement through the dynamic use of a rational
program.  Our specific contributions include:
\begin{enumerate}[(i)]
	\item a technique for devising a mathematical expression in the form of a {\em rational program} to evaluate a performance metric from a set of program and data parameters;
	\item KLARAPTOR, a tool implementing the rational program technique to dynamically optimize {\cuda} kernels by choosing optimal launch parameters; and
	\item an empirical and comprehensive evaluation of our tool on kernels from the \texttt{Polybench/GPU} benchmark suite.
\end{enumerate}

As the KLARAPTOR project is a joint research initiative, my contributions to the project are as follows:
\begin{enumerate}
	\item Profiler Overhaul: I upgraded the profiler to support the latest GPU architectures, 
    transitioning from compute capability 6.0 to 7.5. This update enables KLARAPTOR to target a broader 
    range of NVIDIA GPUs and makes the tool more relevant to contemporary GPU computing environments.

	\item Outlier Removal Process: I introduced a preprocessing step that incorporates an outlier 
    removal algorithm based on quartile fencing. This method helps mitigate the impact of potential noise 
    in the empirical data collected from ncu, stemming from factors such as DVFS and other variances in
    GPU performance.

	\item Enhanced Integration of Rational Programs and MWP-CWP Model: I made improvements to the 
    overall integration of the theory of rational programs and the MWP-CWP model.
\end{enumerate}


\section{Structure of the Thesis}
\label{sec:structure}

The remainder of this thesis is organized as follows.
In the Chapter~\ref{ch:background} we provide a comprehensive introduction to the {\cuda} programming model, 
covering its core concepts and memory model. Additionally, we discuss the theorectical performance model, MWP-CWP,
which is fundamental to our research.  
We introduce the theory
and terminology behind \textit{rational programs} and describe that technique which applies
the idea of rational programs to the goal of dynamically selecting optimal program parameters 
in Chapter~\ref{ch:foundations}.
Chapter~\ref{ch:overview} gives on overview of the KLARAPTOR tool
which applies our technique to {\cuda} kernels and
the general algorithm underlying our tool, that is, 
building and using a rational program to predict program performance.
Then, Chapter~\ref{ch:implementation} describes our
specialization and implementation of this technique to CUDA programs,
while our implementation is evaluated
in Chapter~\ref{ch:performance}.
Lastly, we draw conclusions and explore future work in Chapter~\ref{ch:conclusion}.


\iffalse


Programming for high-performance parallel computing is a notoriously
difficult task. Programmers must be conscious of many factors impacting
performance including scheduling, synchronization, and data locality.
Of course, program code itself impacts the program's performance, however, 
there are still further \textit{parameters} which are independent from the code
and greatly influence performance. 
%
For parallel programs three types of parameters influence performance:
\begin{inparaenum}[(i)]
	\item {\em data parameters}, such as input data and its size;
	\item {\em hardware parameters}, such as cache capacity and number of 
	available registers; and
	\item {\em program parameters}, such as granularity of tasks and 
	the quantities that characterize how tasks are
	mapped to processors (e.g. dimension sizes of a thread block for a {\cuda} kernel).
\end{inparaenum}

Data and hardware parameters are independent from program parameters
and are determined by the needs of the user and available hardware
resources.  Program parameters, however, are intimately related to
data and hardware parameters. The choice of program parameters can
largely influence the performance of a parallel program, resulting in 
orders of magnitude difference in timings (see Section~\ref{sec:performance}).
Therefore, it is crucial to determine values of program
parameters that yield the best program performance for a given set of
hardware and data parameter values.

In the CUDA programming model the kernel launch parameters,
and thus the size and shape of thread blocks, greatly impact performance.
This should be obvious considering that the memory accesses pattern of threads
in a thread block can depend on the block's dimension sizes. 
The same could be said about multithreaded programs on CPU where
parallel performance depends on task granularity and number of threads.
Our general technique (see Section~\ref{sect:steps}) 
is applied on top of some performance model to estimate 
program parameters which optimize performance. 
This could be applied to parallel programs in general, 
where performance models using program parameters exist.
However, we dedicate this paper to the discussion of 
GPU programs written in {\cuda}.
%where optimizing performance is much more difficult.


An important consequence of the impact of kernel launch parameters on performance
is that an optimal thread block format (that is, dimension sizes)
for one GPU architecture may 
not be optimal for another, as illustrated
in~\cite{DBLP:journals/tjs/TorresGL13}.
This emphasizes not only the impact of hardware parameters on program parameters, 
but also the need for performance portability.
That is to say, enabling users to efficiently execute the same parallel program
on different architectures that belong to the same hardware platform.
%\toremove{For example, consider NVIDIA GPUs which have different architectures but all 
%follow the {\cuda} programming model.}

\begin{figure*}[hbt]
\includegraphics[width=1.00\textwidth, trim={0em, 0.9em, 0em, 0em}, clip]{Figures/MinMaxTimes-RTX-8192.png}
\caption{Comparing kernel execution time (log-scaled) for the thread block configuration chosen by KLARAPTOR
versus the minimum and maximum times as determined by an exhaustive search over all possible configurations. 
Kernels are part of the \texttt{PolyBench/GPU} benchmark suite and executed on (1) a RTX 2070 SUPER with a data 
size of $N=8192$ (except convolution3d with $N=512$).}
\label{fig:kernelperfintro}
\end{figure*}

In this thesis, we describe the development of KLARAPTOR (Kernel LAunch
parameters RAtional Program estimaTOR), a tool
for automatically and
dynamically determining the values of  {\cuda} kernel launch parameters which
optimize the kernel's performance, for each kernel invocation
independently.  That is to say, based on the actual data and target
device of a kernel invocation.
%
The accuracy of KLARAPTOR's prediction is illustrated in
Figure~\ref{fig:kernelperfintro} where execution times are given
for each kernel in the the PolyBench/GPU benchmark suite \cite{polybenchGPUweb}
on \reworked{the Turing} architecture.
For each kernel, execution times are shown for 
three different thread block configurations:
one chosen by KLARAPTOR, 
one resulting in the minimum time, and one resulting in the maximum time.
The latter two are decided by an exhaustive search. 
In most cases, KLARAPTOR's prediction is very close to optimal; 
notice that the y-axis is log scaled.
Further experimental results are reported in Section~\ref{sec:performance}.
%We note that in the cases where KLARAPTOR underperforms,
%those kernels are very small, each taking less than 5ms. 
%The source code of KLARAPTOR is freely available at
%\textcolor{navy}{\href{https://github.com/orcca-uwo/KLARAPTOR}{https://github.com/orcca-uwo/KLARAPTOR}}.



KLARAPTOR applies to {\cuda} a generic technique,
also described herein in Section~\ref{sect:steps}, 
to statically build a so-called \toremove{\textit{rational program}}
% ${\cal R}$
which is then used dynamically at runtime to determine optimal program parameters
for a given multithreaded program on specific data and hardware parameters.
%${\cal P}$.
%The goal of our technique and tool are as follows. 
%Let ${\cal E}$ be a performance metric for ${\cal P}$ that we want to
%optimize.  Given particular values of the data parameters
%we look to find values of the program parameters 
%such that the execution of ${\cal P}$ optimizes ${\cal E}$.  
The key principle is based on an observation of most performance metrics.
In most performance prediction models, \textit{high-level performance metrics}, such
as execution time, memory consumption, and hardware occupancy, 
can be seen as decision trees or flowcharts
based on \textit{low-level performance metrics}, such as 
memory bandwidth and cache miss rate.
%, and the number of clock cycles per instruction.
These low-level metrics are themselves
piece-wise rational functions (PRFs) of
program, data, and hardware parameters.
This construction could be applied
recursively to obtain a PRF for the high-level metric.
\fixed{We regard a computer program that computes such a PRF as a {\em rational program}}{I do not feel that there is a logical flow here},
a technical notion defined in Section~\ref{sec:foundations}.

If one could determine these PRFs, 
then it would be possible to estimate, for example, 
the running time of a program based on its
program, data, and hardware parameters. 
Unfortunately, exact formulas for low-level metrics are often not known, 
instead estimated through empirical measures or
assumptions, or collected by profiling.
This is a key challenge our technique addresses.
%For a fixed machine, the hardware parameters are also fixed
%and performance metrics depend only on program and data parameters.  
%

In most cases the values of the data parameters are only given at
runtime, making it difficult to determine optimal values of the
program parameters at an earlier stage. On another hand, a bad choice
of program parameters can have drastic consequences.  Hence, it is
crucial to be able to determine the optimal program parameters at
runtime without much overhead added to the program execution. This is
precisely the intention of the approach proposed here. %in this paper.

Typically, thread block configurations are determined through heuristics 
and are constrained to be multiples of 32 \cite{cuda2016best}. However, our tool utilizes rational 
programs to dynamically optimize thread block configurations based on the size 
of data being operated on, resulting in improved performance for specific kernel 
invocations on target architectures.

KLARAPTOR consists of two main steps: (1) at compile-time, rational programs 
estimating low-level performance metrics are generated for each kernel and are 
inserted into the host code of a CUDA program, and (2) at runtime, the helper 
program corresponding to a kernel is evaluated with specific data parameters to 
determine the thread block configuration that minimizes the kernel's execution time. 
KLARAPTOR also aims to improve programmer performance by providing a more efficient 
alternative to trial and error or exhaustive search.

To minimize compilation and execution overheads, KLARAPTOR analyzes kernel performance 
on small data sizes and maintains a runtime invocation history. Evaluation using the 
Polybench/GPU benchmark suite demonstrates that KLARAPTOR accurately predicts the optimal 
or near-optimal thread block configuration see Figure~\ref{fig:kernelperfintro}.

%Here, optimizing could mean maximizing, as in the case of a
%performance metric such as hardware occupancy, or minimizing, as in the case
%of a performance metric like execution time.
%To address our goal, we compute a mathematical expression, parameterized by data and program parameters,
%in the format of a {\em rational program} ${\cal R}$ at compile-time. 
%At runtime, given the specific values of the data parameters, we can efficiently estimate ${\cal E}$ 
%using ${\cal R}$ and various potential program parameters in order to determine
%the program parameters which optimize ${\cal E}$.


\subsection{Contributions}

The goal of this work is to determine values of program parameters which
optimize a multithreaded program's performance.  Towards that goal,
the method by which such values are found must be
receptive to changing data and changing  hardware parameters. Our contributions
encapsulate this requirement through the dynamic use of a rational
program.  Our specific contributions include:
\begin{enumerate}[(i)]
	\item a technique for devising a mathematical expression in the form of a {\em rational program} to evaluate a performance metric from a set of program and data parameters;
	\item KLARAPTOR, a tool implementing the rational program technique to dynamically optimize {\cuda} kernels by choosing optimal launch parameters; and
	\item an empirical and comprehensive evaluation of our tool on kernels from the \texttt{Polybench/GPU} benchmark suite.
\end{enumerate}

\fi



\section{Related Works}

\label{sec:relatedworks}
%While the idea and the technique described herein are
%applicable to parallel programs in general, this document extensively
%uses GPU architecture and the CUDA (Compute Unified Device Architecture,  %see~\cite{Nickolls:2008:SPP:1365490.1365500,cuda2015})
%programming model for ease of illustration and description.
The {\em Parallel Random Access Machine}
(PRAM) model~\cite{DBLP:journals/siamcomp/StockmeyerV84,Gibbons:1989:MPP:72935.72953},
including models tailored to GPU code analysis
such as 
TMM~\cite{DBLP:journals/fgcs/MaAC14} and  MCM~\cite{DBLP:conf/parco/HaqueMX15}
analyze the performance of parallel programs at an abstract level.
More detailed GPU performance models are proposed such as MWP-CWP 
\cite{DBLP:conf/isca/HongK09,DBLP:conf/ppopp/SimDKV12}, which estimates the execution time of GPU kernels based on the profiling information of the kernels. 


In the context of improving {\cuda} program performance, other research groups have
used techniques such as loop transformation \cite{DBLP:conf/ics/BaskaranBKRRS08},
auto-tuning~\cite{grauer2012auto,Khan:2013:SAC:2400682.2400690,DBLP:books/daglib/p/SatoTKK10,DBLP:conf/icpp/KurzakTGAD19},
dynamic instrumentation~\cite{kistler2003continuous}, or
a combination of the latter two~\cite{song2015automated}.
%%
Auto-tuning techniques have achieved great results
in projects such as 
ATLAS~\cite{DBLP:conf/ppsc/WhaleyD99},
FFTW~\cite{DBLP:conf/icassp/FrigoJ98}, and
SPIRAL~\cite{DBLP:journals/ijhpca/PuschelMSXJPVJ04}
in which multiple kernel versions are generated 
\textit{off-line} and then applied and refined  \textit{on-line}
once the runtime parameters are known.
In contrast, our technique does not optimize the parallel code itself, 
only the program parameters controlling it.

Although much research has been devoted to compiler optimizations 
for kernel source code or PTX code, previous works such
as~\cite{DBLP:conf/cascon/ChenCKMX15} and~\cite{DBLP:journals/tjs/TorresGL13}
suggest that kernel launch
parameters (i.e. thread block configurations)
have a large impact on performance and must be considered
as a target for optimization.
%%
%%
In \cite{liu2009cross}, the authors present an input-adaptive GPU code
optimization framework G-ADAPT, which uses statistical learning to
find a relation between the input sizes and the thread block sizes.
\fixed{At linking time}{Marc: Why can't we be more precise?}, the framework predicts the best block size
for a given input size using the linear model obtained from compile
time.  This approach only considers the total size of the thread
blocks and not their configuration.  Meanwhile, the authors
of~\cite{DBLP:books/daglib/p/SatoTKK10} use a linear regression model
to predict optimal thread block configurations (that is, dimension
sizes and not just the total size). However, they assume kernel execution time
scales linearly with data size.
The authors in~\cite{DBLP:conf/icpp/LimNM17}
have also developed a method determining the best thread block configuration, 
but similarly, they assume execution time scales linearly with data size.
%the authors claim to have a static analysis method which does not
%require running the code for determining the best configuration for
%{\cuda} kernels. 
%Through static code analysis, IPC (instructions per clock-cycle)
%is estimated, however, 
%%They analyze the assembly code generated by
%%the {\texttt{NVCC}} compiler and estimate the IPC %(instruction per clock-cycle)
%%of each instruction, but,
%there is no analysis of memory access patterns, 
%\todo{and it is assumed that kernel execution time is proportional to 
%	the input problem size.}{Alex: I have softened the language this. It was once: 
%	The authors assume 
%	``the execution time of a {\cuda} program is proportional to the
%	input problem size $N$'' \cite[pp.527]{DBLP:conf/icpp/LimNM17}.
%	This may not be practical for some applications such as matrix
%	multiplication.}
%This is, obviously, a very strong assumption, and impractical
%for real world applications where even simple operations such as
%matrix multiplication are not proportional to their input size.
%Moreover, the paper only reports on 4 test-cases 
%(unlike a standard test suite such as PolyBench which includes 15 examples).
%%
%
%Moreover, only two benchmarks are provided.
%%
In~\cite{DBLP:conf/icpp/GarveyA15},
machine learning techniques are used in combination with auto-tuning
to search for optimal configurations of OpenCL kernels, but
their examples are limited to stencil computations. On the other hand, 
ISAAC~\cite{10.1145/3126908.3126939}, an auto-tuning framework, utilizes 
predictive modeling techniques and a regression model on input characteristics 
to generate optimized hardware and application-specific CUDA kernels from 
parameterized PTX code templates. However, its functionality is still restricted 
to matrix multiplication and convolution operations.
%% , but they note many important factors, 
%% namely, high-performing configurations are hard to find and there are very few of them.
%They suggest that the optimization space is
%large but most of them perform poorly. So they apply a
%machine learning model based on the data loading technique
%to pruning the configuration space as the first step. Then two alternative heuristics are used for grouping the remaining optimizations. 




\iffalse
\subsection{Structure of the Thesis}

The remainder of this thesis is organized as follows.
Section~\ref{sec:foundations} formalizes and exemplifies the notion of rational programs
and their relation to piece-wise rational functions and performance prediction.
Section~\ref{sec:klaraptor} gives on overview of the KLARAPTOR tool
which applies our technique to {\cuda} kernels.
The general algorithm underlying our tool, that is, 
building and using a rational program to predict program performance, 
is given in Section ~\ref{sect:steps}.
Our implementation is detailed in Section~\ref{sec:implementation},
while our implementation is evaluated
in Section~\ref{sec:performance}.
Lastly, we draw conclusions and explore future work in Section~\ref{sec:conclusion}.

%We shall see in Section~\ref{sec:embodiment} that ${\cal R}$ can be
%built when the program ${\cal P}$ is being compiled, that is,
%at \textit{compile-time}.  Then, when ${\cal P}$ is executed on a
%specific input data set, that is, at \textit{runtime},
%the rational program ${\cal R}$ can be used
%to make an optimal choice for the program parameters of ${\cal P}$.
%The embodiment of the propped techniques, as reported in 
%Section~\ref{sec:embodiment}, targets Graphics Processing Units (GPUs).
\fi
