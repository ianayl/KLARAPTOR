
\chapter{Experimentation}
\label{ch:performance}



\iffalse 
In this section we highlight the performance and experimentation of
our technique applied to example {\cuda} programs
from the PolyBench benchmarking suite \cite{grauer2012auto}.
For each PolyBench example,
we collect statistical data (see Section~\ref{sec:embodiment}) with
various input sizes $N$ (powers of 2 typically between 64 and 512). 
These are relatively small sizes to allow for fast data collection.
%The results of the emulation are collected, the resulting data is used
The collected data is used 
to produce the rational functions and, lastly, the rational functions
are combined to form the rational programs.
Table~\ref{tab:example_names} gives a short description for
each of the benchmark examples, as well as an index for each kernel
which we use for referring to that kernel in the experimental results.
%We refer to the kernels with their indexes in the following results.

The ability of the rational programs we generate to effectively
optimize the program parameters of each example {\cuda} program is
summarized in Tables~\ref{tab:sanitycheck_nvprof}
and \ref{tab:proofofconcept_nvprof}. The definitions of the notations used
in these tables are as follows:
\begin{itemizeshort}
	\item $N$: the input size of the problem, usually a power of 2,
	\item $Ec$: estimated execution time of the kernel measured in clock-cycles,
	\item $C_r$: best thread block configuration as decided by the {\em rational program},
	\item $C_d$: default thread block configuration given by the benchmark,
	\item $Ec_r$: $Ec$ of configuration $C_r$ as decided by the {\em rational program},
	\item $C_i$: best thread block configuration as decided by \textit{instrumentation},
	\item $Ec_i$: $Ec$ of configuration $C_i$ as decided by \textit{instrumentation},
	\item {\rm Best\_time (B\_t)}: the best {\cuda} running time 
	(i.e. the time to actually run the code on a GPU) of the kernel among all possible configurations,
	\item {\rm Worst\_time (W\_t)}: the worst {\cuda} running time of the kernel among all possible configurations.
\end{itemizeshort}


To evaluate the effectiveness of our rational program, we compare 
the results obtained from it with an actual analysis of the program 
using the performance metrics collected by the instrumentor. 
This comparison gives us estimated clock cycles as shown in 
Table~\ref{tab:sanitycheck_nvprof}.
%%
The table shows the best configurations ($C_i$ and
$C_r$) for $N = 256$ and $N = 512$ for each example, as decided from
both the data collected during instrumentation and the resulting rational
program.  This table is meant to highlight that the construction of
the rational program from the instrumentation data is performed
correctly. Here, either the configurations should be the same or, if
they differ, then the estimates of the clock-cycles should be very
similar; indeed, it is possible that different thread block
configurations (program parameters) result in the same execution
time. Hence, the rational program must choose one of the possibly many
optimal configurations. Moreover, we supply the value ``Collected
$Ec$'' which gives the $Ec$ given by the instrumentation for the thread
block configuration $C_r$.  This value shows that the rational program
is correctly calculating $Ec$ for a given thread block
configuration. The values ``Collected $Ec$'' and $Ec_r$ should be very
close.

Fig.~\ref{fig:atax_kernel2-512}
illustrates detailed results for Kernel 8.2 when $N = 512$. 
The ``\cuda'' plot shows the real runtime of the kernel launched
on a device while the ``RP-Estimated'' plot shows the estimates
obtained from our rational program.
The $x$ axis shows the various thread block configurations.
The runtime values on $y$ axis are normalized to values between 
$0$ and $100$.  

Table~\ref{tab:proofofconcept_nvprof} shows the best configuration ($C_r$)
for $N = 1024$ and $N = 2048$ estimated by the rational program of
each example.  Notice these values of $N$ are much larger than those
used during the instrumentation and construction of the rational
program. This shows that the rational program can extrapolate on the
data used for its construction and accurately estimate optimal program
parameters for different data parameters.  To compare the thread block
configuration deemed optimal by the rational program ($C_r$) and the
optimal thread block configuration determined by actually executing
the program with various thread block configurations we calculate the
value named ``Error''.  This is a  percentage  given
by ($C_rt$ - Best\_time)/(Worst\_time - Best\_time) *
100\%, where ``$C_rt$'' is the actual running time of the program
using the thread block configuration $C_r$ in millisecond. Best\_time  and Worst\_time
are as defined above.  The error percentage shows the
quality of our estimation; it clearly shows the difference between the
estimated configuration and the actual best configuration.  This
percentage should be as small as possible to indicate a good
estimation. 

As an exercise in the practicality of our technique,
 we also show the default thread block configuration ($C_d$) 
 given by the benchmarks  and ``$C_dt$'', the actual running time of the program
using the thread block configuration $C_d$ in the same table. 
Compared to the default configurations, the rational program determines a better configuration
for more than half of the kernels.
Notice that, in most cases, the same default configuration is used 
for all kernels within a single example.
In contrast, our rational program can produce different configurations optimized
for each kernel individually.

As mentioned in Section~\ref{sec:data_collection}, we use
both Ocelot \cite{DBLP:conf/IEEEpact/DiamosKYC10} and
NVIDIA's {\nvprof}\cite{nvprofguide} as instrumentors in the data collection step.
The ``Error''  of the implementation based on data collected by Ocelot is shown in the column ``Ocel'' in Table~\ref{tab:proofofconcept_nvprof}.
We can see that the overall performance is better with \nvprof,
but for some benchmarks, for example Kernel 12,
the best configuration picked by the rational program 
is not as expected. We attribute this to the limitations of the
underlying MWP-CWP model we use.

%\todo{
More experimental results may be found on our GitHub repository:
%of the paper:
%\footnote
{
\small
\color{navy}
\url{https://github.com/orcca-uwo/Parametric_Estimation/tree/master/plots}
}
%}{REWORK}
\fi

In this section we examine the performance of KLARAPTOR 
by applying it to the {\cuda} programs of the 
\texttt{Polybench/GPU} benchmark suite \cite{grauer2012auto}.
We note here that many of the kernels in this
suite perform relatively low amounts of work;
they are best suited to being executed many times
from a loop in the host code.
%Moreover, performance is further varied by dynamic
%frequency and voltage scaling 
%%during benchmarks
%despite setting the performance mode to peak performance.
Data in this section was collected using a RTX 2070 SUPER. 

\begin{table}
\centering
\scriptsize
\caption{KLARAPTOR vs. exhaustive search for thread block configuration choice for kernels in \texttt{Polybench/GPU}.}\label{table:polybenchConfigCompare}
\setlength{\tabcolsep}{1em}
\renewcommand{\arraystretch}{0.9}
\begin{tabular}{lr|rr|rr}
	\toprule
	\multirow{2}{*}{Kernel} & \multirow{2}{*}{N}  &  KLARAPTOR  &  Chosen & Optimal & Optimal$\quad\ $ \\
	  &  & Time (ms) & Config. & Time (ms) & Config. \\
	\midrule
atax K1 
& 2048 &      0.404 &   8, 4 &       0.386 &     16, 2 \\
& 4096 &      0.856 &   16, 2 &      0.856 &     16, 2 \\
& 8192 &      3.175 &   16, 2 &      2.646 &     32, 1 \\
atax K2 
& 2048 &      0.586 &   8, 4 &       0.583 &     16, 2 \\
& 4096 &      1.198 &   32, 1 &      1.196 &     64, 1 \\
& 8192 &      3.096 &   32, 1 &      2.499 &     256, 1 \\
bicg K1 
& 2048 &      0.598 &   8, 4 &      0.598 &     8, 4 \\
& 4096 &      1.217 &  32, 1 &      1.212 &     64, 1 \\
& 8192 &      2.886 &  32, 1 &      2.568 &     32, 2 \\
bicg K2 
& 2048 &      0.388 &   8, 4 &       0.376 &     16, 2 \\
& 4096 &      0.838 &   16, 2 &      0.838 &     16, 2 \\
& 8192 &      3.855 &   16, 2 &      3.042 &    32, 1 \\
convolution2d 
& 2048 &      0.183 &   16, 2 &      0.139 &     32, 4 \\
& 4096 &      0.517 &   32, 2  &     0.517 &     32, 2 \\
& 8192 &      2.265 &   16, 32 &     2.033 &     32, 4 \\
corr 
& 2048 &      1173.972 &   2, 16 &      1173.972 &     2, 16 \\
& 4096 &   10286.199 &   2, 16 &   5799.606 &     16, 2 \\
& 8192 &  366938.750 &   8, 64 &  28347.246 &     32, 1 \\
covar 
& 2048 &      1243.915 &   8, 4 &      1179.058 &     2, 16 \\
& 4096 &   12050.661 &  64, 4 &   5831.021 &     16, 2 \\
& 8192 &  225534.281 &  128, 4 &  27667.753 &     32, 1 \\
fdtd\_step1 
& 2048 &      63.408 &   32, 2 &      63.310 &     256, 1 \\
& 4096 &      250.627 &  32, 2 &      250.302 &    128, 1 \\
& 8192 &      997.219 &  32, 2 &      996.090 &     16, 8 \\
fdtd\_step2 
& 2048 &      91.518 &   32, 1 &      63.998 &     128, 1 \\
& 4096 &      251.701 &  32, 4 &      250.677 &    128, 1 \\
& 8192 &      1385.221 &  16, 2 &     996.072 &    16, 8 \\
fdtd\_step3 
& 2048 &      96.936 &   32, 1 &      84.321 &     32, 2 \\
& 4096 &      330.271 &  32, 2 &      330.259 &    128, 1 \\
& 8192 &      1526.632 &  16, 2 &     1311.767 &   16, 8 \\
gemm 
& 2048 &      143.621 &   16, 2 &      62.054 &     16, 64 \\
& 4096 &    1455.516 &  4, 16 &    491.389 &    16, 64 \\
& 8192 &   12334.238 &  4, 16 &   3940.204 &    16, 64 \\
gesummv 
& 2048 &      1.321 &   4, 8 &      0.773 &     16, 2 \\
& 4096 &      1.818 &   16, 2 &     1.789 &     128, 1 \\
& 8192 &     6.778 &  16, 2 &     5.765 &     32, 1 \\
gramschmidt K1 
& 2048 &      82.855 &   1024, 1 &      81.918 &     4, 16 \\
& 4096 &      370.100 &   1024, 1 &     366.439 &    32, 2 \\
& 8192 &      1511.951 &   1024, 1 &    1494.540 &   32, 2 \\
gramschmidt K2 
& 2048 &      8.581 &   8, 4 &      8.245 &     64, 1 \\
& 4096 &      17.738 &   16, 2 &     17.516 &   32, 1 \\
& 8192 &      38.508 &  32, 1 &      38.364 &   64, 1 \\
gramschmidt K3 
& 2048 &      2227.957 &   32, 1 &   2204.946 &    4, 8 \\
& 4096 &      10020.573 &  32, 1 &   10020.573 &     32, 1 \\
& 8192 &      43879.894 &  32, 1 &   43879.894 &     32, 1 \\
%mean & 4096 &      1.08 &  256, 1 &      1.07 &     32, 1 \\
%& 8192 &      3.79 &  256, 4 &      2.28 &     64, 1 \\
mm2 K1 
& 2048 &      67.812 &   16, 8 &      61.852 &     16, 64 \\
& 4096 &   489.912 &  16, 64 &    489.912 &    16, 64 \\
& 8192 &   3925.331 &  16, 64 &   3925.331 &    16, 64 \\
mm2 K2 
& 2048 &      67.827 &   16, 8 &      61.847 &     16, 64 \\
& 4096 &    488.863 &  16, 64 &    488.863 &    16, 64 \\
& 8192 &   3924.813 &  16, 64 &   3924.813 &    16, 64 \\
mm3 K1 
& 2048 &      67.833 &   16, 8 &      61.840 &     16, 64 \\
& 4096 &    489.801 &  16, 64 &    489.801 &    16, 64 \\
& 8192 &   3926.057 &  16, 64 &   3926.057 &    16, 64 \\
mm3 K2 
& 2048 &      67.905 &   16, 8 &      61.802 &     16, 64 \\
& 4096 &    489.185 &  16, 64 &    489.185 &    16, 64 \\
& 8192 &   3925.962 &  16, 64 &   3925.962 &    16, 64 \\
mm3 K3 
& 2048 &      67.611 & 16, 8 &     61.555 &     16, 64 \\
& 4096 &    489.250 &  16, 64 &    489.250 &    16, 64 \\
& 8192 &   3926.108 &  16, 64 &   3926.108 &    16, 64 \\
mvt K1 
& 2048 &      0.404 &   8, 4 &     0.382 &     16, 2 \\
& 4096 &      0.856 &   16, 2 &     0.856 &    16, 2 \\
& 8192 &     3.298 &  16, 2 &      2.647 &     32, 1 \\
mvt K2 
& 2048 &      0.585 &   8, 4 &      0.584 &    16, 2 \\
& 4096 &      1.198 &  32, 1 &      1.194 &    64, 1 \\
& 8192 &      2.487 &  32, 1 &      2.487 &    32, 1 \\
%reduce & 4096 &      0.01 &  256, 1 &      0.01 &   1024, 1 \\
%& 8192 &      0.01 &  256, 1 &      0.01 &   1024, 1 \\
%std & 4096 &      1.17 &   32, 4 &      1.08 &     32, 1 \\
%& 8192 &     20.67 &   1, 32 &      2.31 &     64, 1 \\
syr2k 
& 2048 &      244.153 &   8, 64 &      242.817 &     8, 4 \\
& 4096 &   1942.944 &   8, 128 &   1941.899 &     8, 64 \\
& 8192 &  21648.312 &  8, 4 &  15545.357 &      8, 128 \\
syrk 
& 2048 &      143.478 &   8, 8 &      143.442 &    8, 64 \\
& 4096 &   1137.063 &   8, 128 &   1136.299 &    8, 64\\
& 8192 &  9085.384 &  8, 128 &   9085.384 &    8, 128 \\
\bottomrule
\end{tabular}
\end{table}

Table~\ref{table:polybenchConfigCompare} 
provides experimental data for the main kernels in the 
benchmark suite \texttt{Polybench/GPU}. Namely, this table compares 
the execution times of the thread block configuration chosen by KLARAPTOR against
the optimal thread block configuration found though exhaustive search. 
The table shows a couple of data sizes in order to highlight that the best configuration 
can change for different input sizes.
While it may appear for some examples that there are large variations 
between timings of the KLARAPTOR-chosen configuration and the optimal, 
these should be considered within the full range of possible configurations.
Recall from Figure~\ref{fig:kernelperfintro} that compared to
the worst possible timings, the KLARAPTOR-chosen configuration 
and the optimal result in very similar in timings.

In Table~\ref{table:opttimetable} we report the time it takes for 
KLARAPTOR to perform its compile-time analysis and build the rational programs
for each example in the \texttt{PolyBench/GPU} suite. This table also 
compares the times 
taken by KLARAPTOR and the exhaustive search.
Exhaustive search times are given as a sum over all possible configurations
and all powers of 2 up to N=8192, 
meanwhile the data collection for KLARAPTOR is done for $128 \leq N \leq 2048$, 
again all powers of 2 within this range.
Note that KLARAPTOR can determine near-optimal thread block configurations
for any $N \geq 128$. As seen in Table~\ref{table:opttimetable}, $\infty$ represents 
the upper bound of KLARAPTOR's search space. 
%%
The best and worst execution times for the main kernel in each example
(for $N=8192$) is also given to highlight the fact that our optimization step
is sometimes faster than even a single execution of a kernel
with a poor choice of thread block configuration.
We note that for some kernels, with very quick running times,
exhaustive search is not a bad option. However, some examples such as
\texttt{GRAMSCHMIDT}, take an exorbitant amount of time for exhaustive search.
This table also shows that the one-time compile-time 
cost can often be amortized by only a few executions of the kernel.

\begin{table}[htb]
    \centering
    \scriptsize
    \caption{KLARAPTOR Optimization Times on \texttt{Polybench/GPU}, RTX 2070 SUPER
    Comparing times for (1) compile-time optimization steps of KLARAPTOR, (2) exhaustive search over all 
    thread block configurations, the execution time for a kernel given (3) the best thread block configuration, and (4) 
    the worst thread block configuration. Exhaustive search is given as a sum for values up to $N=8192$ (except convolution3d with $N=512$).}\label{table:opttimetable}
    \setlength{\tabcolsep}{0.6em}
    \renewcommand{\arraystretch}{1.25}
    \begin{tabular}{l|rr|rr}
        \toprule
        Kernel &  KLARAPTOR Time (s) &  Ex. Search Time (s) &  Min Time (s) &  Max Time (s) \\
        & $128 \leq N<\infty$ & $128 \leq N \leq 8192$ & $N=8192$ & $N=8192$\\
        \midrule
            2DCONV &              210.29 &                       82.78 &      0.002 &         0.023 \\
            ATAX &              507.59 &                       59.60 &        0.006 &         1.940 \\
                MVT &              508.03 &                       60.03 &     0.005 &         1.978 \\
            BICG &              510.91 &                       60.16 &        0.006 &         2.050 \\
            GESUMMV &              398.54 &                      142.78 &     0.006 &         0.129 \\
            GEMM &              456.50 &                      987.77 &        3.941 &       126.052 \\
            SYRK &              579.84 &                     2772.64 &        9.069 &       160.944 \\
            SYR2K &             1173.68 &                     9553.64 &       15.534 &       459.169 \\
                2MM &              700.49 &                     1889.62 &     7.851 &       240.828 \\
                3MM &              944.54 &                     2798.12 &     11.779 &       361.310 \\
            CORR &             1032.92 &                    10924.12 &        28.365 &       861.289 \\
            COVAR &             1141.45 &                    23251.12 &       27.670 &      3900.855 \\
            3DCONV &              132.88 &                       52.06 &      0.006 &         0.053 \\
        GRAMSCHM &             2113.27 &                    94206.06 &        45.418 &     35146.314 \\
            FDTD\_2D &              489.21 &                      495.79 &     3.304 &        21.107 \\
        \bottomrule
    \end{tabular}
\end{table}

%\onecolumn
%\input{plots/2dconv_512}
%\twocolumn



%\input{nvprof_sanity_check_gtx1080ti}
%\input{nvprof_proof_of_concept_gtx1080ti}


%\input{tables/ocelot_error_rate_gtx1080ti}



