\section{An integer hull algorithm}
\begin{frame}{Example (0/3)}
	\begin{block}{Input}
		Let's look at a simple example first.
		
		Vertices: \textcolor{blue}{$(-44/5,408/25),(349/27,206/27),(85/57,109/57)$}
		\begin{columns}
			\begin{column}{0.2\textwidth}
				\begin{center}
					\begin{equation*}
						\color{blue}
						\left\{   \begin{array}{crl}
							2x + 5y &\le& 64\\ 
							7x + 5y &\ge& 20\\
							3x - 6y &\le& -7
						\end{array}
						\right.
					\end{equation*}
					\smallskip
					
					
				\end{center}
				
			\end{column}
			\hfill
			\begin{column}{0.8\textwidth}
				\begin{center}
					\begin{figure}[H]
						\scalebox{.7}{\input{figures/example0}}
					\end{figure}
					
				\end{center}
				
			\end{column}
		\end{columns}
		
		%%
	\end{block}
	
	
\end{frame}
\begin{frame}{Example (1/3)}
	\begin{block}{Normalization}
		Replace the facets that could not have integer point
		
		Vertices: \textcolor{blue}{$(-44/5,408/25),$}\sout{\textcolor{red}{$(349/27,206/27),$$(85/57,109/57),$}} \textcolor{blue}{$(113/9,70/9),$$(25/19,41/19)$}
		\begin{columns}
			\begin{column}{0.2\textwidth}
				\begin{center}
					\sout{\color{red}$3x - 6y \le -7$}
					\begin{equation*}
						\color{blue}
						\left\{   \begin{array}{crl}
							2x + 5y &\le& 64\\ 
							7x + 5y &\ge& 20\\
							3x - 6y &\le& -9
						\end{array}
						\right.
					\end{equation*}
					\smallskip
					
					
				\end{center}
				
			\end{column}
			\hfill
			\begin{column}{0.8\textwidth}
				\begin{center}
					
					\begin{figure}[H]
						\scalebox{.7}{\input{figures/example1}}
					\end{figure}
					
				\end{center}
				
			\end{column}
		\end{columns}
		
		%%
	\end{block}
	
	
\end{frame}
\begin{frame}{Example (2-1/3)}
	\begin{block}{Partition}
		Vertices: $\textcolor{blue}{(-44/5,408/25),(113/9,70/9),(25/19,41/19)}$
		
		Find the triangles with vertices: [$\textcolor{green!60!black}{(-8,16),(-44/5,408/25),(-5,11)}$],
		[$\textcolor{green!60!black}{(3,3),(25/19,41/19),(0,4)}$],
		[$\textcolor{green!60!black}{(12,8),(113/9,70/9),(11,7)}$]
		\begin{columns}
			\begin{column}{0.2\textwidth}
				\begin{center}
					\begin{equation*}
						\color{blue}
						\left\{   \begin{array}{crl}
							5y &\le& -2x + 64\\ 
							5y &\ge& -7x + 20\\
							2y &\ge& x + 3
							
						\end{array}
						\right.
					\end{equation*}
					
				\end{center}
				
			\end{column}
			\hfill
			\begin{column}{0.8\textwidth}
				\begin{center}
					\begin{figure}[H]
						\scalebox{.7}{\input{figures/example2d_1}}
					\end{figure}
				\end{center}
			\end{column}
		\end{columns}
		
		%%
	\end{block}
	
	
\end{frame}
\begin{frame}{Example (2-2/3)}
	\begin{block}{}
		Compute the integer hull of the triangle
		
		Vertices: $\textcolor{red}{(-8,16),(-7,14),(-5,11)}\textcolor{blue}{,(113/9,70/9),(25/19,41/19)}$
		\begin{columns}
			\begin{column}{0.2\textwidth}
				\begin{center}
					\begin{equation*}
						\color{blue}
						\left\{   \begin{array}{crl}
							5y &\le& -2x + 64\\ 
							5y &\ge& -7x + 20\\
							2y &\ge& x + 3
							
						\end{array}
						\right.
					\end{equation*}
					\smallskip
					
					\begin{equation*}
						\color{red}
						\left\{   \begin{array}{crl}
							y &\ge& -2x\\
							2y &\ge& 3x + 7
						\end{array}
						\right.
					\end{equation*}
					
				\end{center}
				
			\end{column}
			\hfill
			\begin{column}{0.8\textwidth}
				
				
				\begin{center}
					\begin{figure}[H]
						\scalebox{.7}{\input{figures/example2}}
					\end{figure}
					
				\end{center}
				
			\end{column}
		\end{columns}
		
		%%
	\end{block}
	
	
\end{frame}

\begin{frame}{Example (3/3)}
	\begin{block}{Merging}
		Vertices: $\textcolor{red}{(-8,16),(-7,14),(-5,11),(0,4),(1,3),(3,3),(11,7),(12,8)}$
		\begin{columns}
			\begin{column}{0.2\textwidth}
				\begin{center}
					\begin{equation*}
						\color{blue}
						\left\{   \begin{array}{crl}
							5y &\le& -2x + 64\\ 
							5y &\ge& -7x + 20\\
							2y &\ge& x + 3
						\end{array}
						\right.
					\end{equation*}
					\smallskip
					
					
					\begin{equation*}
						\color{red}
						\left\{   \begin{array}{crl}
							y &\ge& -2x\\
							2y &\ge& 3x + 7\\
							y &\ge& -x + 4\\
							y &\ge& 3\\
							y &\ge& x - 4
						\end{array}
						\right.
					\end{equation*}
					
				\end{center}
				
			\end{column}
			\hfill
			\begin{column}{0.8\textwidth}
				
				
				\begin{center}
					\begin{figure}[H]
						\scalebox{.7}{\input{figures/example4}}
					\end{figure}
					
				\end{center}
				
			\end{column}
		\end{columns}
		
		%%
	\end{block}
	
	
\end{frame}
\begin{frame}{Main steps of our algorithm}
	Our algorithm has 3 main steps:
	\begin{itemize}
		\item \textbf{Normalization}: during this step, we construct a new polyhedral set
		$Q$ from $P$ as follows.
		Consider in turn each facet $F$ of $P$:
		\begin{enumerate}
			\item if the hyperplane $H$ supporting $F$ contains an integer point,
			then $H$ is a hyperplane supporting a facet of $Q$,
			\item otherwise 
			one slides $H$ towards the center of $P$
			along the normal vector of $F$, stopping 
			as soon as one hits a hyperplane $H'$ containing an integer point,
			then making $H'$ a hyperplane supporting a facet of $Q$.
		\end{enumerate}
		The resulting polyhedral set $Q$ clearly has the same integer
		hull as $P$; computing $Q$ is a preparation phase for the following step.
		\item \textbf{Partitioning}: make each part
		of the partition is a polyhedron $R$ which:
		\begin{enumerate}
			\item either has integer points as vertices
			(making the computation of the integer hull $R_I$ trivial), 
			\item or has a small volume so that any algorithm
			(including exhaustive search) can be applied to compute $R_I$.
		\end{enumerate}
		\item \textbf{Merging}: Once the integer hull of each part of the partition
		is computed and given by the list of its vertices, 
		an algorithm for computing the convex hull of a set points,
		such as {\tt QuickHull},
		can be applied to deduce $P_I$.
	\end{itemize}
\end{frame}
\begin{frame}{The general algorithm}
	\begin{block}{Normalization}
		The integer hull of the normalized polyhedral set should be the same as that of the input
		\begin{columns}
		\begin{column}{0.5\textwidth}
			\begin{equation*}
				\scriptsize
				\label{eq:3dexample}
				\left\{   \begin{array}{crl}
					-98877 x_1 - 189663 x_2 - 1798 x_3 &\le& 705915\\	
					-10109 x_1 - 5958 x_2 - 14601 x_3 &\le& 31333\\
					-5405 x_1 + 4965 x_2 + 3870 x_3 &\le& 4303504\\
					729 x_1 - 117 x_2 + 350 x_3 &\le& 4561\\
					677 x_1 + 465 x_2 - 540 x_3 &\le& 3489
				\end{array}
				\right.
			\end{equation*}
			\begin{center}
				\begin{figure}[H]
					\includegraphics[scale=0.15]{figures/3dexample0}
				\end{figure}	
			\end{center}
		\end{column}
		\hfill
		\begin{column}{0.5\textwidth}
			\begin{equation*}
				\scriptsize
				\label{eq:3dexamplenew}
				\left\{   \begin{array}{crl}
					-98877 x_1 - 189663 x_2 - 1798 x_3 &\le& 705915\\	
					-10109 x_1 - 5958 x_2 - 14601 x_3 &\le& 31333\\
					\color{red}{-1081 x_1 + 993 x_2 + 774 x_3} &\color{red}\le& \color{red}860700\\
					729 x_1 - 117 x_2 + 350 x_3 &\le& 4561\\
					677 x_1 + 465 x_2 - 540 x_3 &\le& 3489
				\end{array}
				\right.
			\end{equation*}
			\begin{center}
				\begin{figure}[H]
					\includegraphics[scale=0.15]{figures/3dexample1}
				\end{figure}
				
			\end{center}
			
		\end{column}
	\end{columns}
		
	\end{block}
\end{frame}
\begin{frame}{The general algorithm}
	\begin{block}{Partition}
		For each face $f$ , 
		\begin{itemize}
			\item let $F$ be a set of all the facets that intersect at $f$
			\item tf there exist integer points on $f$ (which
			means the closest integer points on f to its vertices exist), for each vertex $v$ of $f$, a ``corner'' polyhedral set has candidates of vertices of
			\begin{itemize}
				\item : $v$, 
				\item all the existing closest integer points on $F$
				to $v$,
				\item the closest integer point on $f$ to $v$.
			\end{itemize}
			\item if there is no integer point on $f$, the ``corner''
			polyhedral set has has candidates of vertices of: 
			\begin{itemize}
				\item all the vertices $V$ of $f$,
				\item all the closest integer points on $F$ to $V$.
			\end{itemize}
		\end{itemize}
	\end{block}
\end{frame}

\begin{frame}{The general algorithm}
	\begin{block}{Partition}
		\begin{figure}[htbp]
			\begin{subfigure}[t]{0.3\textwidth}
				\includegraphics[width=\linewidth]{figures/3dexample3}
			\end{subfigure}\hspace{.6in}
			\begin{subfigure}[t]{0.3\textwidth}
				\includegraphics[width=\linewidth]{figures/3dexample2}
			\end{subfigure}
			
			\begin{subfigure}[t]{0.3\textwidth}
				\includegraphics[width=\linewidth]{figures/3dexample4}
			\end{subfigure}\hspace{.6in}
			\begin{subfigure}[t]{0.3\textwidth}
				\includegraphics[width=\linewidth]{figures/3dexample5}
			\end{subfigure}
			\label{figure}
		\end{figure}
	\end{block}
\end{frame}

\begin{frame}{Developing the 3D algorithm}
	\begin{block}{Merging}
		The integer hull has 139 vertices
		\begin{columns}
			\begin{column}{0.5\textwidth}
				\begin{center}
					
					\begin{figure}[H]
						\includegraphics[scale=0.23]{figures/3dexample9}
					\end{figure}
					
				\end{center}
				
			\end{column}
			\hfill
			\begin{column}{0.5\textwidth}
				\begin{center}
					\begin{figure}[H]
						\includegraphics[scale=0.23]{figures/3dexample8}
					\end{figure}
					
				\end{center}
				
			\end{column}
		\end{columns}
		
	\end{block}
\end{frame}

\begin{frame}{Closest integer points on a face to its vertices}
	\begin{block}{Projection and recursive call}
		For a face $F$, and its vertices $V$:
		\begin{itemize}
			\item make a projection
			$G$ of $F$ using Hermite normal form where $G$ is full dimensional. We also obtain a map $R_F(G) = F$. By using Hermite normal form, we ensure that the integer points in $G$ and the integer points in $F$ have a one-on-one relation.
			\item recursively call our integer hull algorithm to compute the vertices $V'_{I}$ of the integer hull of $G$
			\item compute the vertices $V_I$ of $F$ by $R_F(V'_I) = V_I$
			\item find the closest points in $V_I$ to $V$
		\end{itemize}
	\end{block}
\end{frame}

\begin{frame}{Closest integer points on a face to its vertices}
	\begin{block}{Projection and recursive call}
		\begin{columns}
		\begin{column}{0.5\textwidth}
			{\scriptsize
				\begin{equation*}
					R_F: \left\{ \begin{array}{rcl} 
						x_1 &=& 993 x_1' + 573 x_2' - 67995300\\
						x_2 &=& 1081 x_1' + 623 x_2' - 74020200\\
						x_3 &=& x_2'
					\end{array} \right.
				\end{equation*}
			}
			
			\begin{center}
				\begin{figure}[H]
					\includegraphics[scale=0.15]{figures/3dexample7}
				\end{figure}	
			\end{center}
		\end{column}
		\hfill
		\begin{column}{0.5\textwidth}
			\begin{center}
				\begin{figure}[H]
					\includegraphics[scale=0.2]{figures/3dexample6}
				\end{figure}
				
			\end{center}
			
		\end{column}
	\end{columns}
		
	\end{block}
\end{frame}

\begin{frame}{The {\tt PolyhedralSets:-IntegerHull} command in {\Maple} }
	\begin{center}
		
		\begin{figure}[H]
			\includegraphics[scale=0.3]{figures/xmaple_example}
		\end{figure}
	\end{center}
	
\end{frame}
\begin{frame}{The {\tt PolyhedralSets:-IntegerHull} command in {\Maple} }
	\begin{center}
		
		\begin{figure}[H]
			\includegraphics[scale=0.25]{figures/4dcommand}
		\end{figure}
	\end{center}
	
\end{frame}
\begin{frame}{The {\tt PolyhedralSets:-IntegerHull} command in {\Maple}}
	\begin{center}
		
		\begin{figure}[H]
			\includegraphics[scale=0.25]{figures/3dunbounded}
		\end{figure}
	\end{center}
	
\end{frame}
\begin{frame}{Benchmarks 2D}
	\begin{block}{}
		\begin{itemize} 
			\small
			\item E\&C represent ``enumeration and convex hull'' computation. For the Maple implementation we used {\tt ZPolyhedralSets:-EnumerateIntegerPoints} abd {\tt ComputationalGeometry:-ConvexHull}
			\item {\tt Normaliz} is an open source tool for computations in affine monoids, vector configurations, lattice polytopes, and rational cones.
		\end{itemize}
		\centering
		\begin{table}[H]
			
			\resizebox{\columnwidth}{!}{%
				\begin{tabular}{|l|l|l|l|l|l|l|}
					\hline
					Volume &
					\multicolumn{2}{c|}{27.95} &
					\multicolumn{2}{c|}{111.79} &
					\multicolumn{2}{c|}{11179.32} \\
					\hline
					Algorithm & IntegerHull & E\&C &  IntegerHull & E\&C &  IntegerHull & E\&C  \\
					\hline
					Maple (ms) & 172&  410 & 244& 890 & 159& 58083\\
					\hline
					C/C++ (ms) &0.284 &  0.768 & 0.339& 1.676 & 0.286& 6.883\\
					\hline
					Normaliz (ms) &\multicolumn{2}{c|}{835.730} &
					\multicolumn{2}{c|}{462.116} &
					\multicolumn{2}{c|}{1559.401}\\
					\hline
				\end{tabular}
			}
			\caption{Integer hulls of triangles}
		\end{table}
		
		\begin{table}[H]
			
			\resizebox{\columnwidth}{!}{%
				\begin{tabular}{|l|l|l|l|l|l|l|}
					\hline
					Volume &
					\multicolumn{2}{c|}{58.21} &
					\multicolumn{2}{c|}{5820.95} &
					\multicolumn{2}{c|}{23283.82} \\
					\hline
					Algorithm & IntegerHull & E\&C &  IntegerHull & E\&C &  IntegerHull & E\&C  \\
					\hline
					Maple (ms) & 303 &752 & 275& 31357 & 304& 123159\\
					\hline
					C/C++ (ms) &0.451 &  0.565 & 0.478& 0.657 & 0.396& 0.682\\
					\hline
					Normaliz (ms) &\multicolumn{2}{c|}{2.837} &
					\multicolumn{2}{c|}{1216.238} &
					\multicolumn{2}{c|}{740.559}\\
					\hline
				\end{tabular}
			}
			\caption{Integer hulls of hexagons}
		\end{table}
	\end{block}
\end{frame}

\begin{frame}{Benchmarks 3D}
	\begin{block}{}
		\centering
		\begin{table}[H]
			\resizebox{\columnwidth}{!}{%
				\begin{tabular}{|l|l|l|l|l|l|l|}
					\hline
					Volume &
					\multicolumn{2}{c|}{447.48} &
					\multicolumn{2}{c|}{6991.89} &
					\multicolumn{2}{c|}{55935.2} \\
					\hline
					Algorithm & IntegerHull & E\&C &  IntegerHull & E\&C &  IntegerHull & E\&C  \\
					\hline
					Maple (ms) & 977& 7289 & 1223& 74804&1378&531904\\
					\hline
					C/C++ (ms) &4.488 &  0.826 & 4.615& 0.923 & 4.624& 1.527\\
					\hline
					Normaliz (ms) &\multicolumn{2}{c|}{851.495} &
					\multicolumn{2}{c|}{956.666} &
					\multicolumn{2}{c|}{793.192}\\
					\hline
				\end{tabular}
			}
			\caption{Integer hulls of tetrahedrons (4 vertices, 4 facets and 6 edges)}
		\end{table}
		
	\end{block}
	\begin{block}{}
		\centering
		\begin{table}[H]
			\resizebox{\columnwidth}{!}{%
				\begin{tabular}{|l|l|l|l|l|l|l|}
					\hline
					Volume &
					\multicolumn{2}{c|}{412.58} &
					\multicolumn{2}{c|}{7050.81} &
					\multicolumn{2}{c|}{60417.63} \\
					\hline
					Algorithm & IntegerHull &E\&C &  IntegerHull &E\&C &  IntegerHull & E\&C  \\
					\hline
					Maple (ms) & 1476& 5711 & 1573& 60233&1728&512101\\
					\hline
					C/C++ (ms) &11.049 &  21.235 & 16.001& 145.068 & 23.822& 2082.559\\
					\hline
					Normaliz (ms) &\multicolumn{2}{c|}{7862.109} &
					\multicolumn{2}{c|}{N/A} &
					\multicolumn{2}{c|}{N/A}\\
					\hline
				\end{tabular}
			}
			\caption{Integer hulls of triangular bipyramids
				(5 vertices, 6 facets and 9 edges)}
		\end{table}
		
	\end{block}
	
\end{frame}

