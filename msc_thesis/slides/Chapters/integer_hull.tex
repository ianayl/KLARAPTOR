\section{Pseudo-periodicity of vertices of integer hull}

\begin{frame}{Periodic behavior of the integer hulls of a fix-shaped triangle}
	\begin{block}{}
		\begin{columns}
			\begin{column}{0.5\textwidth}
				Consider the triangle defined by the facets
				\begin{equation}
					\left\{   \begin{array}{crl}
						0 & \le & 2\,x -y \\
						0 & \le & y \\
						-b & \le & -3\,x -y \\
					\end{array}
					\right.
				\end{equation}
				\begin{center}
					
					\resizebox{\textwidth}{!}{\includegraphics{figures/number_of_point_3}}
				
				
				This example has a period of 15
			\end{center}
			\end{column}
		\begin{column}{0.5\textwidth}
			Consider the triangle defined by the facets
			\begin{equation}
				\left\{   \begin{array}{crl}
					0 & \le & 2\,x - y  \\
					0 & \le & y \\
					-b & \le & -2\,x -y\\
				\end{array}
				\right.
			\end{equation}

					\begin{center}
						
						\resizebox{\textwidth}{!}{\includegraphics{figures/number_of_point_2}}
	
				This example has a period of 4
			\end{center}
		\end{column}
		\end{columns}
		
	\end{block}
\end{frame}
\begin{frame}{Periodic behavior of the integer hulls of a fix-shaped polyhedron}
	\begin{block}{Visulization of the above example}
			\begin{figure}[htbp]
			\begin{subfigure}[t]{0.3\textwidth}
				\begin{tikzpicture}[scale=.5]% circle - image 1
					\begin{axis}[
						x tick label style={
							/pgf/number format/.cd,
							precision=0,
						},
						y tick label style={
							/pgf/number format/.cd,
							precision=0,
						},
						axis lines=middle,
						grid=major,
						xmin=0,
						xmax=9,
						ymin=0,
						ymax=8,
						xlabel=$x$,
						ylabel=$y$,
						xtick={0,1,...,9},
						ytick={0,1,...,8}
						]  
						\addplot[domain=-1:11,color=black!30!blue]{2*x};
						\addplot[domain=-1:11,color=black!30!blue]{0};
						\addplot[domain=-1:11,color=black!30!green]{-2*x+10}node[above,pos=0.4,anchor=south, font=\footnotesize, sloped]{$2x + y = 10$};
						\addplot[domain=-1:11,color=black!30!red]{-2*x + 14}node[above,pos=0.5,anchor=south, font=\footnotesize, sloped]{$2x + y = 14$};
						\path[pattern=north west lines, pattern color=green!40]
						(0,0) -- (5,0) -- (3,4) -- (2,4) -- cycle;
						
						\path[pattern=north east lines, pattern color=red!40]
						(0,0) -- (7,0) -- (4,6) -- (3,6) -- cycle;
						
					\end{axis}
				\end{tikzpicture}
			\end{subfigure}\hspace{.6in}
			\begin{subfigure}[t]{0.3\textwidth}
								\begin{tikzpicture}[scale=.5]% circle - image 1
					\begin{axis}[
						x tick label style={
							/pgf/number format/.cd,
							precision=0,
						},
						y tick label style={
							/pgf/number format/.cd,
							precision=0,
						},
						axis lines=middle,
						grid=major,
						xmin=0,
						xmax=9,
						ymin=0,
						ymax=8,
						xlabel=$x$,
						ylabel=$y$,
						xtick={0,1,...,9},
						ytick={0,1,...,8}
						]  
						\addplot[domain=-1:11,color=black!30!blue]{2*x};
						\addplot[domain=-1:11,color=black!30!blue]{0};
						\addplot[domain=-1:11,color=black!30!green]{-2*x+11}node[above,pos=0.4,anchor=south, font=\footnotesize, sloped]{$2x + y = 11$};
						\path[pattern=north west lines, pattern color=green!40]
						(0,0) -- (5,0) -- (5,1) -- (3,5) -- (2,4) -- cycle;
					\end{axis}
				\end{tikzpicture}
			\end{subfigure}
			
			\begin{subfigure}[t]{0.3\textwidth}
								\begin{tikzpicture}[scale=.5]% circle - image 1
					\begin{axis}[
						x tick label style={
							/pgf/number format/.cd,
							precision=0,
						},
						y tick label style={
							/pgf/number format/.cd,
							precision=0,
						},
						axis lines=middle,
						grid=major,
						xmin=0,
						xmax=9,
						ymin=0,
						ymax=8,
						xlabel=$x$,
						ylabel=$y$,
						xtick={0,1,...,9},
						ytick={0,1,...,8}
						]  
						\addplot[domain=-1:11,color=black!30!blue]{2*x};
						\addplot[domain=-1:11,color=black!30!blue]{0};
						\addplot[domain=-1:11,color=black!30!green]{-2*x+12}node[above,pos=0.4,anchor=south, font=\footnotesize, sloped]{$2x + y = 12$};
						\path[pattern=north west lines, pattern color=green!40]
						(0,0) -- (6,0) -- (3,6) -- cycle;
					\end{axis}
				\end{tikzpicture}
			\end{subfigure}\hspace{.6in}
			\begin{subfigure}[t]{0.3\textwidth}
								\begin{tikzpicture}[scale=.5]% circle - image 1
					\begin{axis}[
						x tick label style={
							/pgf/number format/.cd,
							precision=0,
						},
						y tick label style={
							/pgf/number format/.cd,
							precision=0,
						},
						axis lines=middle,
						grid=major,
						xmin=0,
						xmax=9,
						ymin=0,
						ymax=8,
						xlabel=$x$,
						ylabel=$y$,
						xtick={0,1,...,9},
						ytick={0,1,...,8}
						]  
						\addplot[domain=-1:11,color=black!30!blue]{2*x};
						\addplot[domain=-1:11,color=black!30!blue]{0};
						\addplot[domain=-1:11,color=black!30!green]{-2*x+13}node[above,pos=0.4,anchor=south, font=\footnotesize, sloped]{$2x + y = 13$};
						\path[pattern=north west lines, pattern color=green!40]
						(0,0) -- (6,0) -- (6,1) -- (4,5) -- (3,6) -- cycle;
							
					\end{axis}
				\end{tikzpicture}
			\end{subfigure}
		\end{figure}
		
	\end{block}
\end{frame}

\begin{frame}{Angular sector and its integer hull}
	\begin{definition}
		An {\em angular sector} in an affine plane
		is defined by the intersection of two half-planes whose boundaries intersect
		in a single point, called the {\em vertex} of that angular sector.
	\end{definition}
\begin{block}{}
	\begin{figure}[H]
	\begin{subfigure}{.40\textwidth}
	\centering
	%		\includegraphics[width=8cm]{figures/non_coprime1.png}
	\begin{tikzpicture}[scale=0.8]% circle - image 1
	\begin{axis}[my style, ticks=none,minor tick num=0,scale=.7]
		\addplot[domain=-1:7,color=black]{2/3 * x + 2}node[above,pos=0.5,anchor=south, font=\footnotesize, sloped]{$-2x + 3y \le 6$};
		\addplot[domain=-1:7,color=black!30!green]{-x + 10/3}node[below,pos=0.5,anchor=north, font=\footnotesize, sloped]{$-3x - 3y \le -10$};
		\addplot[mark=*,color=black,only marks,mark size=1pt] coordinates {(6,6)}node[pin=0:{$C$}]{} ;
		\addplot[mark=*,color=black!60!green,only marks,mark size=1pt] coordinates {(6,-8/3)}node[pin=180:{$B$}]{} ;
		\addplot[mark=*,color=black!60!green,only marks,mark size=1pt] coordinates {(4/5, 38/15)}node[pin=180:{$A$}]{} ;
	\end{axis}
\end{tikzpicture}
\caption{One way to compute the integer hull of a sector is to find a triangle where two vertices are integer points}
\end{subfigure}
	\begin{subfigure}{.40\textwidth}
	
	\centering
	%		\includegraphics[width=8cm]{figures/non_coprime1.png}
	\begin{tikzpicture}[scale=0.8]% circle - image 1
		\begin{axis}[my style, ticks=none,minor tick num=0,scale=.7]
			\addplot[domain=-1:7,color=black]{2/3 * x + 2}node[above,pos=0.5,anchor=south, font=\footnotesize, sloped]{$-2x + 3y \le 6$};
			\addplot[domain=-1:7,color=black!30!green]{-x + 10/3}node[below,pos=0.5,anchor=north, font=\footnotesize, sloped]{$-3x - 3y \le -10$};
			\addplot[domain=-1:7,color=black!30!red]{-x + 4}node[above,pos=0.5,anchor=south, font=\footnotesize, sloped]{$-x - y \le -4$};
			\addplot[mark=*,color=black,only marks,mark size=1pt] coordinates {(6,6)}node[pin=0:{$C$}]{} ;
			\addplot[mark=*,color=black!60!green,only marks,mark size=1pt] coordinates {(6,-8/3)}node[pin=180:{$B$}]{} ;
			\addplot[mark=*,color=black!60!green,only marks,mark size=1pt] coordinates {(4/5, 38/15)}node[pin=180:{$A$}]{} ;
			\addplot[mark=*,color=black!60!red,only marks,mark size=1pt] coordinates {(6/5,42/15)}node[pin=0:{$A'$}]{} ;
			\addplot[mark=*,color=black!60!red,only marks,mark size=1pt] coordinates {(6,-2)}node[pin=0:{$B'$}]{} ;
		\end{axis}
	\end{tikzpicture}
\caption{The integer hull of sector $BAC$ is the same as that of sector $B'AC'$}
\end{subfigure}
\end{figure}

\end{block}

\end{frame}


\begin{frame}{Periodicity of the integer hull of an angular sector}
\begin{theorem}
	\label{th:angle_period}
	Let us consider a parametric angular sector $S(b_i)$ defined by
	\begin{equation*}
		\left\{   \begin{array}{crl}
			a_1\,x + c_1\,y &\le& b_1  \\
			a_2\,x + c_2\,y &\le& b_2  \\
		\end{array}
		\right.
	\end{equation*}
	where $\gcd{a_i,b_i,c_i} = 1$ for $i \in \{ 1, 2\}$ and $a_i, b_i,c_i$ are all integers, $b_i \in \{b_1,b_2\}$.
	%%	
	Let $S_I(b_i)$ be the integer hull of $S(b_i)$. Then, there exists an integer $T$ and a vector $\vec{u}$  such that $S_I(b_i + T)$ is the translation of $S_I(b_i)$ by $\vec{u}$. 
	
	\[T = \frac{1}{g_2}\,\lvert a_2\,c_1 - a_1\,c_2 \rvert\quad or\quad \frac{1}{g_1}\,\lvert a_2\,c_1 - a_1\,c_2 \rvert\]
	 \[\vec{u} = (\frac{c_2\,T}{a_2\,c_1 - a_1\,c_2}, \frac{a_2\,T}{a_2\,c_1 - a_1\,c_2}) \quad or\quad   (\frac{c_1\,T}{a_1\,c_2 - a_2\,c_1}, \frac{a_1\,T}{a_1\,c_2 - a_2\,c_1})\]
	  for $b_i = b_1$ or $b_i = b_2$ respectively, where $g_i = \gcd{a_i,c_i}$. Note that $a_2\,c_1 - a_1\,c_2 \neq 0$ holds.
\end{theorem}
\end{frame}

\begin{frame}
	\begin{block}{}
		\begin{itemize}
			\item $D$ and $E$ are the closest integer points to $A$ on the two side of sector $BAC$, $D'$ and $E'$ are the closest integer points to $A'$ on the two side of sector $BA'C'$
			\item We can prove that $ \protect\overrightarrow{AE} =\protect\overrightarrow{A'E'}$ and $ \protect\overrightarrow{AD} = \protect\overrightarrow{A'D'}$
			\item For any integer point $F$ in $\triangle ADE$, there is an integer points $F'$ in $\triangle A'D'E'$ such that $\vec{FF'} = \vec{AA'}$ 
		\end{itemize}
			\begin{figure}[H]
			\centering
			\begin{tikzpicture}% circle - image 1
				\begin{axis}[my style, minor tick num=0]
					\addplot[domain=-1:11,color=black]{2/3 * x + 2};
					\addplot[domain=-1:13,color=black!30!green]{-3/5*x + 2};
					\addplot[domain=-1:13,color=black!30!red]{-3/5*x + 29/5};
					\addplot[mark=*,color=black,only marks,mark size=1pt] coordinates {(9.324,8.216)}node[pin=180:{$B$}]{} ;
					\addplot[mark=*,color=black!60!green,only marks,mark size=1pt] coordinates {(10,-4)}node[pin=180:{$C$}]{} ;
					\addplot[mark=*,color=black!60!green,only marks,mark size=1pt] coordinates {(0, 2)}node[pin=180:{$A$}]{} ;
					\addplot[mark=*,color=black!60!red,only marks,mark size=1pt] coordinates {(3,4)}node[pin=180:{$A'$}]{} ;
					\addplot[mark=*,color=black!60!red,only marks,mark size=1pt] coordinates {(10.5,-0.5)}node[pin=180:{$C'$}]{} ;
					\addplot[mark=*,color=black!60!green,only marks,mark size=1pt] coordinates {(1.5,3)}node[pin=180:{$E$}]{} ;
					\addplot[mark=*,color=black!60!red,only marks,mark size=1pt] coordinates {(4.5,5)}node[pin=180:{$E'$}]{} ;
					
					\addplot[mark=*,color=black!60!green,only marks,mark size=1pt] coordinates {(4,-0.4)}node[pin=180:{$D$}]{} ;
					\addplot[mark=*,color=black!60!red,only marks,mark size=1pt] coordinates {(7,1.6)}node[pin=180:{$D'$}]{} ;
					\draw[color=gray] (4,-0.4) -- (1.5,3);	
					\draw[color=gray] (7,1.6) -- (4.5,5);	
				\end{axis}
			\end{tikzpicture}
		\end{figure}
	\end{block}
\end{frame}

\begin{frame}{Periodicity of the integer hull of an 2D-angular sector}
	\begin{block}{}
	\begin{figure}[H]
		\begin{subfigure}{.40\textwidth}
			\centering
			% include first image
			\animategraphics[autoplay,loop,scale=.7]{1}{figures/sector2}{}{}
			\caption{}
		\end{subfigure}
		\begin{subfigure}{.40\textwidth}
			\centering
			\animategraphics[autoplay,loop,scale=.7]{1}{figures/sector2_1}{}{} 
			\caption{}
		\end{subfigure}
		\caption{A more complicated example. The red dots are the vertices of the integer hull of the sector.}
	\end{figure}	
	\end{block}
\end{frame}

\begin{frame}{The Pseudo-periodicity of the Integer Hulls}
	\begin{theorem}
		Let $P(b)$ be a parametric polygon given by 
		\begin{equation}  
			a_i\,x + c_i\,y \le b_i 
		\end{equation}
		where $i\in \{1,\ldots, n\}$ and the parameter $b \in \{b_1,\ldots, b_n\}$ and $P_I(b)$ be the integer hull of $P(b)$. Specifically, $a_i\,x + c_i\,y \le b_i$ and $a_{i+1}\,x + c_{i+1}\,y \le b_{i+1} $ define an angular sector $S_i$ of $P$, we have $i + 1 = 1$ if $i = n$,
		There exists an integer $T$ and $n$ vectors $\vec{v_1}, \ldots, \vec{v_n}$, 
		such that for $\lvert b\rvert $ large enough $P_I(b + T)$  can be obtained from $P_I(b)$ as
		following.
		
		As is defined above, denoting $S_i$ the sectors of $P(b)$ and by $S_{iI}$,
		their respective integer hulls. then
		\[P_I(b+T) = \bigcap f_{v_i}(S_{iI})\]
		where $f_{v_i}$ is the translation by $v_i$.
	\end{theorem}
\end{frame}

\begin{frame}
	Triangle $P(b)$ defined by
	\begin{equation}
		\left\{   \begin{array}{crl}
			-x+4\,y &\ge& 4  \\
			2\,x-y&\ge& 0  \\
			-x -y &\ge& b  \\
		\end{array}
		\right.	
	\end{equation}
	First, we look at the angular sector $S(b)$ given by $-x+4\,y \ge 4$ and $-x -y \ge b$. $S_I( b - 5\,n)$ is a transformation of $S_I(b- 5\,(n-1))$ by $\vec{(4,1)}$ for any $n \ge 1$. 
	
	When $ \lvert b\rvert \ge 11$, the integer hull of $P(b -15n)$ is a translation of $P(b-15(n-1))$ by $\vec{(0,0)},\vec{(12,3)},\vec{(5,10)}$ for $n \ge 1$.
	
	\begin{figure}[H]
		\begin{subfigure}{.42\textwidth}
			\centering
			% include first image
			\animategraphics[controls=all,loop,scale=.7]{1}{figures/sector_animation}{}{} 
			\caption{Integer hull of an angular sector}
			\label{fig:sec_anim}
		\end{subfigure}
		\hfill
		\begin{subfigure}{.42\textwidth}
			\centering
			% include second image
			\animategraphics[controls=all,loop,scale=0.7]{1}{figures/triangle_animation}{}{} 
			\caption{Integer hull of a triangle}
			\label{fig:tri_anim}
		\end{subfigure}
		\caption{The periodic phenomenon in a simple example. The dots are the vertices of the integer hull.}
		\label{fig:simple}
	\end{figure}
\end{frame}


\iffalse
\begin{frame}{Application}
	\begin{block}{Parametric integer hull}
	If the given $P$ is a parametric polyhedron where $b_i$ is unknown, we propose the following steps to compute the vertices of $P_I$:
	\begin{itemize}
		\item compute the length of a cycle $T$ and the transformation vectors.
		\item compute the integer hull of every non-parametric polyhedron in one cycle. 
		\item when the values of the parameters are available, using the corresponding solution from the previous step and the vectors from step 2 to compute the integer hull of the $P$ with the given parameters.
	\end{itemize}
	
	Note that we can finish the first two steps ``off-line'', once the parameters are given the only computation that needs to be done is the translations which could be done in linear time. This method is both time and space efficient if the cycle $T$ is short.
\end{block} 
	
\end{frame}


\begin{frame}{Proposal}
	\begin{block}{What's next?}
		\begin{itemize}
			\item Adapting the theorems to 3D polyhedral set
			\begin{itemize}
				\item Define the equivalent of ``angular sector'' in 3D and find the periodicity of the vertices of its integer hull. \item The difficulty lies in the edge areas. We need to find a way to partition the edge polyhedral sets such that the their volumes would not related to the length of the edges. 
				
				By doing so, we will be able to prove that the number of the vertices of a 3D integer hull does not related to its volume. 
			\end{itemize}
			\item Implement the parametric IntegerHull procedure 
		\end{itemize}
	\end{block}
\end{frame}

\fi