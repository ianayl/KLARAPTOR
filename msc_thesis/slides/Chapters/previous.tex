\section{Literature review}
\begin{frame}{Integer hull computation}
	\begin{block}{The cutting-plane method}
		\begin{itemize}
			\item One important family of algorithms for computing
			$P_I$ relies on the {\bf cutting plane method}
			\item Introduced by Gomory in~\cite{DBLP:books/daglib/p/Gomory10} to solve integer
			linear programming (ILP) and mixed-integer programming (MILP) problems
			\item This method is based on finding a sequence of linear inequalities (cuts) to reduce the feasible region to the original ILP problem. 
			\item Chv\'atal~\cite{DBLP:journals/dm/Chvatal73a} and
			Schrijver~\cite{schrijver1980cutting} gave a geometrical description of the cutting plane method and developed a procedure to
			compute $P_I$ based on it. 
			\item Schrijver gave a full
			proof and a complexity study of this method
			in~\cite{DBLP:journals/networks/Rajan90}.
		\end{itemize}
	\end{block}
\end{frame}

\begin{frame}{The cutting-plane method}
	\begin{block}{Example 1/2}
\begin{equation}
	\begin{array}{ll@{}ll}
		\text{maximize}  & \displaystyle y & &\\
		\text{subject to}& x,y \in \Z &&\\
		&x &\ge 0&\\
		& -3x + 2y &\le 0 &\\
		& 3x + 2y &\le 6&		
	\end{array}
	\label{eq:ilp}
\end{equation}
\begin{figure}[htb]
	\centering % <-- added
	\begin{subfigure}{0.5\textwidth}
		\resizebox{\linewidth}{!}{\input{../figures/cut0}}
		\caption{An integer linear programming problem}
		\label{fig:cut0}
	\end{subfigure}\hfil % <-- added
	\begin{subfigure}{0.5\textwidth}
		\resizebox{\linewidth}{!}{\input{../figures/cut1}}
		\caption{Adding the first cut}
		\label{fig:cut1}
	\end{subfigure}
	\caption{Solving ILP with cutting-plane method}
	\label{fig:cut}
\end{figure}
	\end{block}
\end{frame}

\begin{frame}{The cutting-plane method}
	\begin{block}{Example 2/2}
		\begin{equation}
			\begin{array}{ll@{}ll}
				\text{maximize}  & \displaystyle y & &\\
				\text{subject to}& x,y \in \Z &&\\
				&x &\ge 0&\\
				& -3x + 2y &\le 0 &\\
				& 3x + 2y &\le 6&	
			\end{array}
			\label{eq:ilp}
		\end{equation}
		\begin{figure}[htb]
			\centering % <-- added
			\begin{subfigure}{0.5\textwidth}
				\resizebox{\linewidth}{!}{\input{../figures/cut2}}
				\caption{Adding the second cut}
				\label{fig:cut2}
			\end{subfigure}\hfil % <-- added
			\begin{subfigure}{0.5\textwidth}
				\resizebox{\linewidth}{!}{\input{../figures/cut3}}
				\caption{Optimum integer solution}
				\label{fig:cut3}
			\end{subfigure}
			\caption{Solving ILP with cutting-plane method II}
			\label{fig:cut_1}
		\end{figure}
	\end{block}
\end{frame}

\begin{frame}{Integer hull computation}
	\begin{block}{The branch-and-bound method}
		\begin{itemize}
			\item Another approach for computing $P_I$ uses the {\bf branch-and-bound
				method}, introduced by Land and Doig in the early 1960s
			in~\cite{land1960automatic}.
			\item This method is also first introduced to solve the integer optimization problem and ILP problem
			\item This method recursively divides $P$ into
			sub-polyhedra, then the vertices of the integer hull of each part of the
			partition are computed.
		\end{itemize}
	\end{block}
\end{frame}

\begin{frame}{The branch-and-bound method}
	\begin{block}{Example}
\begin{equation*}
	\begin{array}{ll@{}ll}
		\text{maximize}  & \displaystyle  4x + 5y & &\\
		\text{subject to}&  x,y \in \Z &&\\
		&x, y&\ge 0&\\
		&x + 4y &\le 10 &\\
		& 3x - 4y &\le 6&		
	\end{array}
\end{equation*}
\begin{figure}[H]
	\centering % <-- added
	\begin{subfigure}{0.5\textwidth}
		\resizebox{\linewidth}{!}{\input{../figures/bb1}}
		\caption{Adding the second cut}
		\label{fig:bb1}
	\end{subfigure}\hfil % <-- added
	\begin{subfigure}{0.5\textwidth}
		\resizebox{\linewidth}{!}{\input{../figures/bb2}}
		\caption{Optimum integer solution}
		\label{fig:bb2}
	\end{subfigure}
	\caption{Solving ILP with cutting-plane method II}
	\label{fig:bb}
\end{figure}
	\end{block}
\end{frame}


\begin{frame}{Enumeration or counting the lattice points}
	\begin{block}{Latice point counting}
		\begin{itemize}
			\item Pick's theorem:

		\item Barvinok~\cite{DBLP:journals/mor/Barvinok94} algorithm: 	counting the integer points inside a polyhedron, which runs in
		polynomial time, for a fixed dimension of the ambient space.
		
		\end{itemize}
	\begin{figure}[H]
		\centering % <-- added
		\begin{subfigure}{0.25\textwidth}
						\begin{figure}[H]
				\includegraphics[scale=0.3]{figures/pick}
			\end{figure}
			\caption{Pick's theorem}
		\end{subfigure}\hfil % <-- added
		\begin{subfigure}{0.75\textwidth}
				\begin{figure}[H]
				\includegraphics[scale=0.3]{figures/barvinok}
			\end{figure}
			\caption{An example for Barvinok decomposition}
		\end{subfigure}
	\end{figure}
	\end{block}
\end{frame}


\begin{frame}{Enumeration or counting the lattice points}
	\begin{block}{Latice point counting}
		\begin{itemize}
			\item In 2004, the software package {\tt LattE}~\cite{DBLP:journals/jsc/LoeraHTY04} offers the first implementation of Barvinok's algorithm.
			\item Verdoolaege etc.
			present in~\cite{DBLP:journals/algorithmica/VerdoolaegeSBLB07}
			a novel method for lattice point counting, based on Barvinok's
			decomposition.
			\item Yanagisawa~\cite{yanagisawa2005simple} gave a simpler approach for lattice
			point counting, which divides a polygon into {\bf right-angle
				triangles} and rectangles then calculates the number of lattice points within 
			each such triangle.
		\end{itemize}
	\end{block}
\end{frame}


\begin{frame}{Enumeration or counting the lattice points}
	\begin{block}{Lattice point enmulation}
		\begin{itemize}
			\item Jing and Moreno Maza~\cite{DBLP:conf/casc/JingM17} present a algorithm that compute an irredundant representation of the integer points of $P$ in terms of lower-dimensional polyhedral sets.
			\item {\tt Normaliz}~\cite{bruns2010normaliz} is a program for the computation of Hilbert basis of rational cones. 
			 Let $C \in \R^d$ convex cone  and $L \in \Z^d$ be a lattice. There exists a unique minimal generating set, $H = {x_1,\ldots,x_n}$, of $ C \cap L$, such that every point $x \in C \cap L$ has an integer conical combination:
			\[x = \lambda_1x_1 + \cdots + \lambda_nx_n, \lambda_1,\ldots,\lambda_n \in \Z, \lambda_1,\ldots,\lambda_n \ge 0\]
		\end{itemize}
		\begin{figure}[H]
		\centering % <-- added
			\begin{figure}[H]
				\includegraphics[scale=0.7]{figures/HB}
			\end{figure}
			\caption{Pick's theorem}
	\end{figure}
	\end{block}
\end{frame}

\begin{frame}{Vertices of Integer Hull}
	\begin{block}{Number of vertices}
		\begin{itemize}
			\item The earlier study by Cook,
			Hartmann, Kannan and McDiarmid, shows that the number of vertices of $P_I$ is related to 
			the {\textit{size}} of the coefficients of the inequalities that describe $P$.
			\item $x = p/q$ is a rational number, $p$ and $q$ are coprime
			\item $size(x) = 1 + \lceil(log(\lvert p\rvert + 1))\rceil + \lceil(log(\lvert q\rvert + 1))\rceil$
			\item The size of a linear inequality with coefficient vector $\vec{a} = (a_1,\ldots,a_n)$ is 
			\[size(\vec{a}) = n + size(a_1) + \ldots + size(a_n)\]
			\item For a polyhedron $P = \{\x \ | \ A\x \leq \b \}$ where matrix $A \in \Q^{m \times n}$ and vector $\b \in \Q^m$, $\varphi$ is the maximum size of the $m$ inequalities, then the number of vertices of $P_I$ is at most $2m^n(6n^2\varphi)^{n-1}$. 	
		\end{itemize}
		
	\end{block}
\end{frame}

\begin{frame}{Parametric integer hull}
	\begin{block}{}
		\begin{itemize}
			\item Parametric polyhedral set, particularly we define
			$P(\b) = \{\x \ | \ A\x \leq \b \}$ where $\b$ is unknown at compile time, either the whole vector is unknown or some element of the vector is unknown, we can represent the later case by $P(b_i)$.
		\end{itemize}
		
	\end{block}
	\begin{block}{}
		\begin{columns}
			\begin{column}{0.5\textwidth}
				\begin{center}
						\resizebox{\textwidth}{!}{\includegraphics{figures/parametric.png}}
				\end{center}
			\end{column}
			\begin{column}{0.5\textwidth}
				\begin{center}
					\resizebox{\textwidth}{!}{\includegraphics{figures/number_of_point_3}}
				\end{center}
			\end{column}
		\end{columns}
	\end{block}
\end{frame}

\begin{frame}{``Periodicity'' of $P_I$}
	
	\begin{block}{}
		\begin{itemize}			
			\item For each integer $n \geq 1$,  Eug\`eme Ehrhart defined
			the {\textit{dilation}} of the polyhedron $P$ by $n$ as the
			polyhedron $nP = \{nq \in {\Q}^d \ \mid \ q \in P \}.$
			Ehrhart studied the number of
			lattice points in $nP$, that is:
			\begin{equation*}
				i(P, n) \ \ = \ \ \# (nP  \ \cap \  {\Z}^d ) \ \ = \ \
				\# \{ q \in P \ \mid \  nq \in {Z}^d \}.
			\end{equation*}
			He proved that 
			there exists an integer $N > 0$ and
			polynomials $f_0, f_1 ,  \ldots, f_{N-1}$
			such that $i(P, n) = f_i (n)$ if $n \equiv i \ \mod{N}.$
			The quantity 
			$i(P, n)$ is called the {\textbf{\textit{Ehrhart quasi-polynomial}}} of $P$,
			in the dilation variable $n$.
			
			\item In 2004, Meister presents a new method 
			for computing the integer hull of a parameterized rational polyhedron.
			The author introduces a concept of periodic polyhedron
			(with facets given by equalities depending on
			periodic numbers). Hence, the word ``periodic'' means that the polyhedron 
			can be defined in a periodic manner which is different from our perspective.
		\end{itemize}
		
	\end{block}
\end{frame}